\chapter{Introduction and Motivation} \label{ch:int}
In todays connected world the communication is essential for the proper functionality of most systems. Every smallest peace of electronics is somehow connected to others and with the IoT era it will only grow in popularity. With nowadays technologies many connections can be performed without the use of any physical medium, like wires or fiber-optics. The wireless communication is used not only in cellular networks any more, but more and more frequently in the short range communication or even industrial applications. The communication standards developed up to now are not suitable for this task.

The goal of the ParSec project is to create a dependable, flexible and secure wireless communication system which meets all industrial automation requirements. It has to work with latencies below 1 ms and with very high noise level, serving many distributed clients at once. Moreover it has to deal with fading effects and potentially many reflections or even obstacles coming in the way of transmission and breaking it. The part of the work that has been assigned to the Faculty of Technical Informatics at the Brandenurg University of Technology Cottbus-Senftenberg was to cover the dependability of the wireless communication.

Most of the nowadays error correcting technologies are not capable of correcting multiple bit errors or are too slow to meet the new challenges. The work of P. Pfeifer and H. T. Vierhaus focused on creating a fully configurable and adaptable Forward Error Correction (FEC) mechanism to allow multiple bit error correction in transmission path \cite{art:Pfeifer}. Moreover their work resulted in flexible test strategies and error detection and correction in dependable communication systems \cite{art:Gleichner}. This approach may prove useful only in systems, where the hardware modules are accessible for the test designers. Moreover the test strategies cover only the digital parts of the system. The transmitter and receiver modules are complex systems containing digital, analog and mixed-signal circuitry. With no internal access, due to patent protection, the test may only be conducted based on external interfaces of such modules. The main goal of this very thesis was to implement a diagnostic test strategy for error detection, based on FEC units present in the system. Since the implementation included newly designed encoder and decoder architectures, the test of those two were conducted in passing and summarized. Additionally as a byproduct, a test bench has been built, to allow easy test and design validation. This may be reused in future work, to shorten the development process.

\hyperref[ch:dependability]{Chapter~\ref*{ch:dependability}} describes the issue of dependability and the threats which a modern communication systems need to face. The concept of permanent and transient faults is introduced and some countermeasures discussed. Forward Error Correction is presented as an effective mean to deal with communication errors and some soft errors caused by transient faults in hardware modules lying between the encoder and decoder.

\hyperref[ch:test]{Chapter~\ref*{ch:test}} focuses on testing the hardware. It describes a general approach in testing and explains how tests are conducted and what do they consist of. The final part describes which Design for Testability (DfT) techniques are widely used and what are they limitations. A special emphasis is laid on the faults in FEC encoders and decoders.

\hyperref[ch:test_bench]{Chapter~\ref*{ch:test_bench}} talks about the specially developed test bench used for setting up the environment for the diagnostic test and for testing two other designs used by the ParSec project - the encoder and decoder. The test bench has been redesigned and adapted to the current needs and ended up as an effective tool for test of hardware designs implemented into an FPGA structure. It's a further step than the post-synthesis simulation, allowing to conduct tests on "real" design.

In \autoref{ch:experiment} the implementation of the diagnostic test is presented. The results of FEC BCH Encoder test and BCH Decoder tests for maximal frequency are summarized. Further, the evaluation of errors due to overclocking takes place, followed by fault injection. Finally a systematic test scenario is presented with comments resulting from the work on this thesis.




