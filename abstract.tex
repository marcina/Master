\phantomsection
\addcontentsline{toc}{chapter}{Abstract}
\begin{abstract}
Wireless communication is more and more commonly used in highly unfavorable environments. Nowadays it's not only cellular communication, but also communication within cities, industrial production facilities and other short distance, real-time applications. Such systems have significant demand on safety whilst being vastly exposed not only to signal interferences, multi-path signal propagation and fading effects but also to transient and permanent faults in hardware. While communication errors are covered by increasingly effective error correcting codes, the permanent faults in hardware still pose a threat to dependability. Since the communication systems consist of digital, analog and mixed-signal circuitry, the diagnostic test to uncover permanent faults happens in every module separately. The test extensions have to be built in during the development process, which is difficult in systems with no access to the internal structure of some IPs, e.g. due to patent protection. The following thesis describes the implementation of a diagnostic test, while treating the communication system as a whole and using the forward error correction units for error position determination.
\vfill


 \noindent Marcin Aftowicz m.j.aftowicz@gmail.com \newline


\end{abstract}