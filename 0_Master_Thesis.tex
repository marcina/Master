\documentclass[]{myclass}
\usepackage[cp1250]{inputenc}
\usepackage[OT4]{fontenc}
\usepackage{hyperref}
\usepackage[table,xcdraw]{xcolor}
\usepackage{multirow}
\usepackage{subfig}
\usepackage{float}
\usepackage{amsfonts}
\usepackage{pdfpages}
\usepackage{listings}
\usepackage[english]{babel}
\usepackage{titling}
\usepackage{tocbibind}
\usepackage[all]{nowidow}
\usepackage{comment}
\usepackage{caption}
\usepackage{indentfirst}
\definecolor{mygreen}{rgb}{0,0.6,0}
\lstset {
basicstyle=\small,
breaklines=true,
commentstyle=\color{mygreen},
keywordstyle=\color{blue},
language=C,
linewidth=\textwidth
}
\linespread{1}

\author{Marcin Aftowicz}
\title{Hardware Test and fault diagnosis based on extended FEC functions  in wireless communication systems}
\mysupervisor{Prof. Dr.-Ing. H. T. Vierhaus \and Ing. Petr Pfeifer, MSc, MBA, Ph.D.}
\myyear{2018}

\begin{document}
\selectlanguage{english}
\bibliographystyle{plplain}

% Front matter **************************************
\frontmatter
\pagestyle{empty}%
\maketitle  \cleardoublepage

\pagenumbering{Roman}
\phantomsection
\addcontentsline{toc}{chapter}{Statutory declaration}

{\noindent}STATUTORY DECLARATION\\

I declare that I have authored this thesis independently, that I have not used other than the declared 
sources  /  resources,  and  that  I  have  explicitly  marked  all  material  which  has  been  quoted  either  
literally or by content from the used sources. \\
\par\vspace{15mm}\par

Marcin Aftowicz \hfill Date \hspace{2cm}

   \cleardoublepage

\pagestyle{ppfcmthesis}
\phantomsection
\addcontentsline{toc}{chapter}{Abstract}
\begin{abstract}
Created at the end
\vfill


 \noindent Marcin Aftowicz m.j.aftowicz@gmail.com \newline


\end{abstract}    \cleardoublepage

\listoffigures  \cleardoublepage
\listoftables   \cleardoublepage

\hypersetup{
    linkcolor={blue!70!black},
    citecolor={blue!70!black},
    urlcolor={blue!70!black}
}
\tableofcontents \cleardoublepage

% Main matter **************************************
\mainmatter
 
%Define necessary functions for diagnostic tests	   
%Define encoder/decoder extensions	   
%Implement extensions on FPGAs	   
%Develop diagnostic test interface software to control / set optional parameters and record results	   
%Implement into FPGA an experimental set-up	   
%Conduct measurements on examples

%das Block-Diagramm zu ParSec zeigen
%zeigen wie der Baseband-Prozessor aufgebaut ist (digital-analog), so dass ein direkter interner Test mit bekannten "digitalen" Verfahren schwierig ist.
%vorstellen, dass man bei einem System wie in ParSec ohne FEC nicht auskommt.
%Zeigen, dass FEC-Komponenten partiell eigene Fehler korrigieren.
%zeigen (Petr!) wie man bei den FEC-Komponenten Selbsttest-Funktionen einbauen kann.

%=============================================================
\input{"1_introduction.tex"}
%=============================================================
\input{"2_dependability"}
%=============================================================
\input{"3_test.tex"}
%=============================================================
\input{"4_experiment.tex"}
%=============================================================
\input{"5_test_bench.tex"}

% All appendices and extra material, if you have any.
\cleardoublepage
\appendix%

%\chapter{Rysunki techniczne}
\begin{figure} [h]
\centering
%%----start of first subfigure----
	\subfloat[G�rna warstwa p�ytki]{\label{fig:subfig:front} 
	\includegraphics[height=0.3\textheight]{figures/Board_PCB_front.JPG}}
	\hfill
%%----start of second subfigure----
	\subfloat[Dolna warstwa p�ytki]{\label{fig:subfig:back}
	\includegraphics[height=0.3\textheight]{figures/Board_PCB_back.JPG}}
	\caption{Model 3D nak�adki na Discovery}
	\label{fig:3D} %% label for entire figure
\end{figure}

\includepdf[trim=0 0 -1cm 0, pages={1}]{figures/Motherboard_bw.pdf}
\includepdf[trim=-1cm 0 0 0, pages={1}]{figures/Bachelor_Hub.pdf}
\noindent



%%\input{plyta.tex}
%\cleardoublepage
\hypersetup{ linkcolor={black}}
\cleardoublepage

\bibliography{bibliografia}
\end{document}